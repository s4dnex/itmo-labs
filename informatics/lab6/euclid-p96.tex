\documentclass{article}
\usepackage[english,russian]{babel}
\usepackage[a5paper, top=1cm, bottom=1.5cm, left=0.5cm, right=1.25cm]{geometry}
\usepackage{amsmath,amssymb} % Математические формулы
\usepackage{paracol} % Разбиение текста на колонки
\usepackage{graphicx} % Для добавления иллюстраций
\usepackage{tikz} % Для рисования фигур
\usepackage{wrapfig} % Для текста вокруг иллюстрации
\usetikzlibrary {angles, calc} % Дополнительные библиотеки для tikz

\graphicspath{{./img/}} % Путь до картинок

\columnratio{0.4,0.6} % Соотношение размеров колонок
\setlength{\emergencystretch}{4em} % Максимальное расстояние между словами
\setlength\parindent{2.5em} % Отступ абзаца
\pagestyle{empty} % Стиль страницы без колонтитулов

\begin{document}

    % Title

    \begin{paracol}{2}
        \begin{column}
        \end{column}

        \begin{column}
            \noindent96\qquad КНИГА III ПРЕДЛ. IV. ТЕОРЕМА\hfill \\
            % \vspace{2em}
        \end{column}
    \end{paracol}

    % Text

    \begin{paracol}{2}
        \begin{column}
            \begin{center}
                \begin{tikzpicture}
                    \coordinate [label=left:\tiny{$A$}] (A) at (-2.04,0.2);
                    \coordinate [label=left:\tiny{$B$}] (B) at (-1.87,-0.84);
                    \coordinate [label=right:\tiny{$C$}] (C) at (1.45,-1.45);
                    \coordinate [label=right:\tiny{$D$}] (D) at (2.01,-0.4);
                    \coordinate [label=left:\tiny{$F$}]  (F) at (0,0);
                    \coordinate (G) at (-0.45,-2);
                    \coordinate [label=below:\tiny{$E$}] (E) at (-0.23,-0.65);
                    % \draw (F) -- (G);
                    \draw pic [fill=blue,angle radius=0.4cm] {angle = D--E--F};
                    \draw pic [fill=yellow,angle radius=0.4cm] {angle = C--E--D};
                    \draw[black, dashed, ultra thick] (F) -- (E);
                    \draw[black, ultra thick] (A) -- (C);
                    \draw[orange, ultra thick] (B) -- (D);
                    

                    \draw[blue, ultra thick] (F) let
                                \p1 = ($ (G) - (F) $)
                                in
                                circle ({veclen(\x1,\y1)});
                \end{tikzpicture}
            \end{center}
        \end{column}

        \begin{column}
            \noindent\begin{minipage}[t][6em]{\linewidth}
                \begin{wrapfigure}{l}{0.2\linewidth}
                    \includegraphics[width=1.7cm]{letter-E.png}
                \end{wrapfigure}
                \textit{\\ сли в круге две прямые, не проходящие через центр, пересекаются, они не делят друг друга пополам.}
            \end{minipage}
            
            \vspace{1em}
            \par Если одна из прямых проходит через центр, очевидно, она ее не может рассекать пополам другая прямая, не проходящая через центр.
            \par Но если ни одна из прямых 
            \begin{tikzpicture}
                \draw[black, ultra thick] (0,0) -- (1,0);
                \node[above] at (0,0) {{\tiny$A$}};
                \node[above] at (1,0) {{\tiny$C$}};
            \end{tikzpicture} 
            или
            \begin{tikzpicture}
                \draw[orange, ultra thick] (0,0) -- (1,0);
                \node[above] at (0,0) {{\tiny$B$}};
                \node[above] at (1,0) {{\tiny$D$}};
            \end{tikzpicture}
            не проходит через центр, проведем 
            \begin{tikzpicture}
                \draw[black,dashed, ultra thick] (0,0) -- (1,0);
                \node[above] at (0,0) {{\tiny$E$}};
                \node[above] at (1,0) {{\tiny$F$}};
            \end{tikzpicture}
            из центра к точке их пересечения.\\
            
            \noindent\begin{minipage}[t][6em]{\linewidth}
                \begin{center}
                    Если
                    \begin{tikzpicture}
                        \draw[black, ultra thick] (0,0) -- (1,0);
                        \node[above] at (0,0) {{\tiny$A$}};
                        \node[above] at (1,0) {{\tiny$C$}};
                    \end{tikzpicture}
                    делится пополам,\\
                    \begin{tikzpicture}
                        \draw[black, dashed, ultra thick] (0,0) -- (1,0);
                        \node[above] at (0,0) {{\tiny$E$}};
                        \node[above] at (1,0) {{\tiny$F$}};
                    \end{tikzpicture}
                    $\bot$ ей (пр. III.$_3$)
                    \[
                    \begin{array}{c@{\hskip 1em}c@{\hskip 1em}c}
                        \therefore &
                        \begin{tikzpicture}[baseline]
                            \coordinate [label=below:\tiny{$C$}] (C) at (0.53,-0.25);
                            \coordinate (D) at (0.27,0.03);
                            \coordinate [label=left:\tiny{$F$}]  (F) at (0.23,0.65);
                            \coordinate [label=left:\tiny{$E$}] (E) at (0,0);
                            \draw pic [fill=blue] {angle = D--E--F};
                            \draw pic [fill=yellow] {angle = C--E--D};
                        \end{tikzpicture} = &
                        \begin{tikzpicture}[baseline]
                            \coordinate (A) at (-0.5, 0);
                            \coordinate (B) at (0, 0);
                            \coordinate (C) at (0, 0.5);
                            \draw [black, thick] (A) -- (B);
                            \draw [black, thick] (B) -- (C);
                            \draw pic [draw, black, ultra thick] {angle = C--B--A};
                        \end{tikzpicture}
                    \end{array}
                    \]
                    и если
                    \begin{tikzpicture}
                        \draw[orange, ultra thick] (0,0) -- (1,0);
                        \node[above] at (0,0) {{\tiny$B$}};
                        \node[above] at (1,0) {{\tiny$D$}};
                    \end{tikzpicture}
                    делится пополам,\\
                    \begin{tikzpicture}
                        \draw[black, dashed, ultra thick] (0,0) -- (1,0);
                        \node[above] at (0,0) {{\tiny$E$}};
                        \node[above] at (1,0) {{\tiny$F$}};
                    \end{tikzpicture}
                    $\bot$
                    \begin{tikzpicture}
                        \draw[orange, ultra thick] (0,0) -- (1,0);
                        \node[above] at (0,0) {{\tiny$B$}};
                        \node[above] at (1,0) {{\tiny$D$}};
                    \end{tikzpicture} (пр. III.$_3$)
                    \[
                    \begin{array}{c@{\hskip 1em}c@{\hskip 1em}c}
                        \therefore &
                        \begin{tikzpicture}[baseline]
                            \coordinate (C) at (0.53,-0.25);
                            \coordinate [label=right:\tiny{$D$}] (D) at (0.36,0.04);
                            \coordinate [label=left:\tiny{$F$}]  (F) at (0.23,0.65);
                            \coordinate [label=left:\tiny{$E$}] (E) at (0,0);
                            \draw pic [fill=blue] {angle = D--E--F};
                            % \draw pic [fill=yellow] {angle = C--E--D};
                        \end{tikzpicture} = &
                        \begin{tikzpicture}[baseline]
                            \coordinate (A) at (-0.5, 0);
                            \coordinate (B) at (0, 0);
                            \coordinate (C) at (0, 0.5);
                            \draw [black, thick] (A) -- (B);
                            \draw [black, thick] (B) -- (C);
                            \draw pic [draw, black, ultra thick] {angle = C--B--A};
                        \end{tikzpicture};
                    \end{array}
                    \]
                    \[
                    \begin{array}{c@{\hskip 1em}c@{\hskip 1em}c}
                        \text{и } \therefore &
                        \begin{tikzpicture}[baseline]
                            \coordinate (C) at (0.53,-0.25);
                            \coordinate [label=right:\tiny{$D$}] (D) at (0.36,0.04);
                            \coordinate [label=left:\tiny{$F$}]  (F) at (0.23,0.65);
                            \coordinate [label=left:\tiny{$E$}] (E) at (0,0);
                            \draw pic [fill=blue] {angle = D--E--F};
                            % \draw pic [fill=yellow] {angle = C--E--D};
                        \end{tikzpicture} = &
                        \begin{tikzpicture}[baseline]
                            \coordinate [label=below:\tiny{$C$}] (C) at (0.53,-0.25);
                            \coordinate (D) at (0.27,0.03);
                            \coordinate [label=left:\tiny{$F$}]  (F) at (0.23,0.65);
                            \coordinate [label=left:\tiny{$E$}] (E) at (0,0);
                            \draw pic [fill=blue] {angle = D--E--F};
                            \draw pic [fill=yellow] {angle = C--E--D};
                        \end{tikzpicture};
                    \end{array}
                    \]
                    часть равна целому, что невозможно.\\
                    \vspace{1em}
                    $\therefore$
                    \begin{tikzpicture}
                        \draw[black, ultra thick] (0,0) -- (1,0);
                        \node[above] at (0,0) {{\tiny$A$}};
                        \node[above] at (1,0) {{\tiny$C$}};
                    \end{tikzpicture}
                    и
                    \begin{tikzpicture}
                        \draw[orange, ultra thick] (0,0) -- (1,0);
                        \node[above] at (0,0) {{\tiny$B$}};
                        \node[above] at (1,0) {{\tiny$D$}};
                    \end{tikzpicture}
                    не делят друг друга пополам.
                    \rightline{ч.т.д.}
                \end{center}
            \end{minipage}
        \end{column}

    \end{paracol}

    \clearpage
\end{document}
